%% For double-blind review submission, w/o CCS and ACM Reference (max submission space)
% \documentclass[acmsmall,review, anonymous]{acmart}\settopmatter{printfolios=true,printccs=false,printacmref=false}
%% For double-blind review submission, w/ CCS and ACM Reference
%\documentclass[acmsmall,review,anonymous]{acmart}\settopmatter{printfolios=true}
%% For single-blind review submission, w/o CCS and ACM Reference (max submission space)
\documentclass[acmsmall,review]{acmart}\settopmatter{printfolios=true,printccs=false,printacmref=false}
%% For single-blind review submission, w/ CCS and ACM Reference
%\documentclass[acmsmall,review]{acmart}\settopmatter{printfolios=true}
%% For final camera-ready submission, w/ required CCS and ACM Reference
%\documentclass[acmsmall]{acmart}\settopmatter{}


%% Journal information
%% Supplied to authors by publisher for camera-ready submission;
%% use defaults for review submission.
\acmJournal{PACMPL}
\acmVolume{1}
\acmNumber{OOPSLA} % CONF = POPL or ICFP or OOPSLA
\acmArticle{1}
\acmYear{2023}
\acmMonth{1}
\acmDOI{10.1145/nnnnnnn.nnnnnnn} % \acmDOI{10.1145/nnnnnnn.nnnnnnn}
\startPage{1}

%% Copyright information
%% Supplied to authors (based on authors' rights management selection;
%% see authors.acm.org) by publisher for camera-ready submission;
%% use 'none' for review submission.
\setcopyright{none}
%\setcopyright{acmcopyright}
%\setcopyright{acmlicensed}
%\setcopyright{rightsretained}
%\copyrightyear{2018}           %% If different from \acmYear

%% Bibliography style
\bibliographystyle{ACM-Reference-Format}
%% Citation style
%% Note: author/year citations are required for papers published as an
%% issue of PACMPL.
\citestyle{acmauthoryear}   %% For author/year citations


%%%%%%%%%%%%%%%%%%%%%%%%%%%%%%%%%%%%%%%%%%%%%%%%%%%%%%%%%%%%%%%%%%%%%%
%% Note: Authors migrating a paper from PACMPL format to traditional
%% SIGPLAN proceedings format must update the '\documentclass' and
%% topmatter commands above; see 'acmart-sigplanproc-template.tex'.
%%%%%%%%%%%%%%%%%%%%%%%%%%%%%%%%%%%%%%%%%%%%%%%%%%%%%%%%%%%%%%%%%%%%%%


%% Some recommended packages.
\usepackage{booktabs}   %% For formal tables:
                        %% http://ctan.org/pkg/booktabs
\usepackage{subcaption} %% For complex figures with subfigures/subcaptions
                        %% http://ctan.org/pkg/subcaption

\usepackage{meu}

\begin{document}

%% Title information
\title{Exploiting Factorization For Scaling Decision-Theoretic Probabilistic Programs}         %% [Short Title] is optional;
                                        %% when present, will be used in
                                        %% header instead of Full Title.
% \titlenote{with title note}             %% \titlenote is optional;
%                                         %% can be repeated if necessary;
%                                         %% contents suppressed with 'anonymous'
% \subtitle{Subtitle}                     %% \subtitle is optional
% \subtitlenote{with subtitle note}       %% \subtitlenote is optional;
%                                         %% can be repeated if necessary;
%                                         %% contents suppressed with 'anonymous'


%% Author information
%% Contents and number of authors suppressed with 'anonymous'.
%% Each author should be introduced by \author, followed by
%% \authornote (optional), \orcid (optional), \affiliation, and
%% \email.
%% An author may have multiple affiliations and/or emails; repeat the
%% appropriate command.
%% Many elements are not rendered, but should be provided for metadata
%% extraction tools.

%% Author with single affiliation.
% \author{First1 Last1}
% \authornote{with author1 note}          %% \authornote is optional;
%                                         %% can be repeated if necessary
% \orcid{nnnn-nnnn-nnnn-nnnn}             %% \orcid is optional
% \affiliation{
%   \position{Position1}
%   \department{Department1}              %% \department is recommended
%   \institution{Institution1}            %% \institution is required
%   \streetaddress{Street1 Address1}
%   \city{City1}
%   \state{State1}
%   \postcode{Post-Code1}
%   \country{Country1}                    %% \country is recommended
% }
% \email{first1.last1@inst1.edu}          %% \email is recommended

% %% Author with two affiliations and emails.
% \author{First2 Last2}
% \authornote{with author2 note}          %% \authornote is optional;
%                                         %% can be repeated if necessary
% \orcid{nnnn-nnnn-nnnn-nnnn}             %% \orcid is optional
% \affiliation{
%   \position{Position2a}
%   \department{Department2a}             %% \department is recommended
%   \institution{Institution2a}           %% \institution is required
%   \streetaddress{Street2a Address2a}
%   \city{City2a}
%   \state{State2a}
%   \postcode{Post-Code2a}
%   \country{Country2a}                   %% \country is recommended
% }
% \email{first2.last2@inst2a.com}         %% \email is recommended
% \affiliation{
%   \position{Position2b}
%   \department{Department2b}             %% \department is recommended
%   \institution{Institution2b}           %% \institution is required
%   \streetaddress{Street3b Address2b}
%   \city{City2b}
%   \state{State2b}
%   \postcode{Post-Code2b}
%   \country{Country2b}                   %% \country is recommended
% }
% \email{first2.last2@inst2b.org}         %% \email is recommended

\author{Minsung Cho}
\affiliation{
  \institution{Northeastern University}            %% \institution is required
  \city{Boston}
  \state{Massachusetts}
  \country{USA}                    %% \country is recommended
}
\email{minsung@ccs.neu.edu}          %% \email is recommended

\author{Steven Holtzen}
\affiliation{
  \institution{Northeastern University}            %% \institution is required
  \city{Boston}
  \state{Massachusetts}
  \country{USA}                    %% \country is recommended
}
\email{s.holtzen@northeastern.edu}          %% \email is recommended

%% Abstract
%% Note: \begin{abstract}...\end{abstract} environment must come
%% before \maketitle command
\begin{abstract}
Abstract TBW
\end{abstract}


%% 2012 ACM Computing Classification System (CSS) concepts
%% Generate at 'http://dl.acm.org/ccs/ccs.cfm'.
\begin{CCSXML}
<ccs2012>
<concept>
<concept_id>10011007.10011006.10011008</concept_id>
<concept_desc>Software and its engineering~General programming languages</concept_desc>
<concept_significance>500</concept_significance>
</concept>
<concept>
<concept_id>10003456.10003457.10003521.10003525</concept_id>
<concept_desc>Social and professional topics~History of programming languages</concept_desc>
<concept_significance>300</concept_significance>
</concept>
</ccs2012>
\end{CCSXML}

\ccsdesc[500]{Software and its engineering~General programming languages}
\ccsdesc[300]{Social and professional topics~History of programming languages}
%% End of generated code


%% Keywords
%% comma separated list
\keywords{}  %% \keywords are mandatory in final camera-ready submission


%% \maketitle
%% Note: \maketitle command must come after title commands, author
%% commands, abstract environment, Computing Classification System
%% environment and commands, and keywords command.
% \maketitle

\begin{definition}
  A \textbf{branch-and-bound algebra} is a semiring $\mathcal S = (S, + , \times, 0 ,1)$ equipped with orders $(\leq, \sqsubseteq)$ such that:
  \begin{enumerate}
    \item $(S, \leq)$ is a total order,
    \item $(S, \sqsubseteq)$ is a join--semilattice with join $\sqcup$ where $\sqsubseteq$ respects $+,\times$, 
    \item $\leq, \sqsubseteq$ are \textit{compatible} in the sense that for all $a \sqsubseteq b$, we also have $a \leq b$.
  \end{enumerate}
\end{definition}

\begin{example}
  The nonnegative real numbers $\R^{\geq 0}$ forms a branch-and-bound algebra with the usual semiring structure of $\R^{\geq 0}$ and the usual order serving as both $\leq$ and $\sqsubseteq$, with join being the $\max$ function.
\end{example}

\begin{example}
  The Boolean semiring $\mathbb B = \{\top, \bot\}$ with $+ = \lor, \times = \land, 0 = \bot, 1 = \top$ forms a branch-and-bound algebra with the order $\bot \leq \top$ with join $\land$. 
\end{example}


\begin{example}
  The expected utility semiring $\R^{\geq 0} \times \R$ with the usual semiring operations forms a branch-and-bound algebra with:
  \begin{enumerate}
    \item $(p,u) \leq (q,v)$ iff $u \leq v$ or $u=v$ and $p \leq q$ 
    \item $(p,u) \sqsubseteq (q,v)$ iff $p \leq q$ and $u \leq v$, with join being a coordinatewise max.
  \end{enumerate} 
  It is straightforward to see that these are compatible.
\end{example}

\begin{example}
  For any branch-and-bound algebra $\mathcal B = (\mathcal B, 0, 1, +, \times, \leq, \sqsubseteq)$ consider the collection of finite sets with elements in $\mathcal B$, $\mathcal P_{<\omega} (\mathcal B)$. This forms a semiring with additive and multiplicative identities $\{0\}, \{1\}$ with:
  \begin{enumerate}
    \item $A + B = \cup_{A,B} \{a+b\}$,
    \item $A \times B = \cup_{A,B} \{a \times b\}$.
  \end{enumerate}
  Moreover it becomes a branch-and-bound algebra with:
  \begin{enumerate}
    \item $A \leq B$ iff $\max A \leq \max B$, where $\max$ is the greatest in the set with respect to $\leq$, 
    \item $A \sqsubseteq B$ iff for all $a \in A$ there exists $b \in B$ with $a \sqsubseteq b$, with join 
    \begin{equation}
      A \sqcup B = \cup_{A,B} \begin{cases}
        a & a \sqcup b = a \\ 
        b & a \sqcup b = b \\
        \{a,b\} & else.
      \end{cases}
    \end{equation}
  \end{enumerate}
\end{example}

The intuition here is that $\leq$ is a total order that allows for a selection between "fully evaluated" values and $\sqsubseteq$ is a partial order that allows for comparisons between "partially evaluated" values. The compatibility condition is effectively saying that "comparable partially evaluated values will stay comparable once fully evaluated".

\begin{definition}
  Let $X, D$ be disjoint sets of variables and let $\varphi$ a formula with variables $X \cup D$. Let $\pi(X), \pi(D)$ denote assignments to $X,D$ respectively. Fix $\mathcal B$ a branch-and-bound algebra; let $f : \mathcal P X \times \mathcal P D \to \mathcal B$ a function; the associated \textbf{max-sum} problem is
  \begin{equation}
    \max_{x \in \pi(X)} \sum_{y \in}
  \end{equation}
\end{definition}

%% Acknowledgments
% \begin{acks}                            %% acks environment is optional
%                                         %% contents suppressed with 'anonymous'
%   %% Commands \grantsponsor{<sponsorID>}{<name>}{<url>} and
%   %% \grantnum[<url>]{<sponsorID>}{<number>} should be used to
%   %% acknowledge financial support and will be used by metadata
%   %% extraction tools.
%   This material is based upon work supported by the
%   \grantsponsor{GS100000001}{National Science
%     Foundation}{http://dx.doi.org/10.13039/100000001} under Grant
%   No.~\grantnum{GS100000001}{nnnnnnn} and Grant
%   No.~\grantnum{GS100000001}{mmmmmmm}.  Any opinions, findings, and
%   conclusions or recommendations expressed in this material are those
%   of the author and do not necessarily reflect the views of the
%   National Science Foundation.
% \end{acks}


%% Bibliography
% \bibliography{meu.bib}


% Appendix

\end{document}
